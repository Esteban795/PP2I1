\documentclass[10pt,a4paper]{report}
\usepackage{geometry}
\geometry{hmargin=2.5cm,vmargin=1.5cm}

\usepackage{wrapfig}

\usepackage[utf8]{inputenc}
\usepackage{amsmath, amsfonts, amssymb}

\usepackage[dvipsnames]{xcolor}
\colorlet{darkblue}{blue!60!black}
\colorlet{darkgreen}{green!60!black}

\usepackage{graphicx} %Loading the package
\graphicspath{{data/}} %Setting the graphicspath

\newcommand{\tabitem}{~~\llap{\textbullet}~~}

\usepackage{hyperref}

\usepackage{tcolorbox}
\usepackage{ulem}

\usepackage{titlesec}

\usepackage{hyperref}



\usepackage{subcaption}
\captionsetup{compatibility=false}


\usepackage{array} %%% Permet de faire des tabular mieux
\usepackage{longtable} %%%permet des tableaux sur plusieurs pages


\newcommand{\imp}[0]{$\implies$}

%%%%%%%%%%%%%%%%%%%%%%%%%%%%%%%%%%% mise en page des fonction section, sub, .... %%%%%%%%%%%%%%%%%%%%%%%%%%%%%%%%%%%
\titleformat{\chapter}[display]
{\normalfont\huge\bfseries}{Chapitre \thechapter\ :}{20pt}{\Huge}

\titleformat{\part}[display]
{\Huge\scshape\filright}
{THEME \thepart}
{20pt}
{\thispagestyle{plain}}
\makeatother

\titleformat{\section}
{\normalfont\Large\bfseries}{\thesection}{1em}{}

\titleformat{\subsection}
{\normalfont\large\bfseries}{\thesubsection}{1em}{}

\titleformat{\subsubsection}
{\normalfont\normalsize\bfseries}{\thesubsubsection}{1em}{}

\renewcommand{\thepart}{MO\arabic{chapter}\_\Roman{part}}
\makeatletter

\renewcommand{\thechapter}{MO\arabic{chapter}}
\makeatletter

\renewcommand{\thesection}{\Roman{section}.}

\renewcommand{\thesubsection}{\Roman{section}.\arabic{subsection}.}
\makeatletter

\renewcommand{\thesubsubsection}{MO\arabic{chapter}\_ \Roman{section}.\A{subsection}.\alph{subsubsection}.}
\makeatletter
%%%%%%%%%%%%%%%%%%%%%%%%%%%%%%%%%%%%%%%%%%%%%%%%%%%%%%%%%%%%%%%%%%%%%%%%%%%%%%%%%%%%%%%%%%%%%%%%%%%%%%%%%%%%%%%%%%%%

\newcommand{\linesep}[0]{\begin{center}
                        \rule{0.90\linewidth}{0.15pt}
                        \end{center}}


%\usepackage[titles]{tocloft}
%\renewcommand*\cftchapnumwidth{3em}
%\renewcommand*\cftsecnumwidth{4.3em}
%\renewcommand*\cftsubsecnumwidth{5.3em}


%%%%%%%%%%%%%%%%%%%%%%%%%%%%%%%%%%% Set-up fonction encadrement %%%%%%%%%%%%%%%%%%%%%%%%%%%%%%%%%%%


\newtcolorbox{Encadrer}{colback=white,
colframe=black}

\newtcbox{\encadrer}{colframe=black, colback=white, nobeforeafter,box align=base,size=fbox}

\newtcbox{\encasumup}{colframe=black, colback=white, nobeforeafter,box align=base,size=fbox, left=6pt,right=6pt}

\newtcolorbox{sumup}{colback=white, colframe=black, fonttitle=\bfseries, title={test}}

\usepackage[tikz]{bclogo}
\renewcommand\bcStyleTitre[1]{
\smash{\raisebox{1em}{\colorbox{white}{\color{black}\encasumup{ Résumé }}}}}%

\newtcolorbox{definition}[2][]{%
colback=green!5!white,colframe=green!55!black,fonttitle=\bfseries,
title={Definition} \thetcbcounter: #2,#1}

\newtcolorbox{important}{colback=white,
colframe=red!55!black}


%%%%%%%%%%%%%%%%%%%%%%%%%%%%%%%%%%% Set-up header and foot %%%%%%%%%%%%%%%%%%%%%%%%%%%%%%%%%%%
\usepackage{fancyheadings}
\pagestyle{plain}
%%%%%%%%%%%%%%%%%%%%%%%%%%%%%%%%%%%%%%%%%%%%%%%%%%%%%%%%%%%%%%%%%%%%%%%%%%%%%%%%%%%%%%%%%%%%%%

\title{PPII \\ -- \\ Compte Rendu 6}
\date{18 / 12 / 2023}
\author{BS}

\begin{document}
\maketitle

\tableofcontents \bigskip

\noindent\rule{\linewidth}{0.5mm} \bigskip

                \begin{tabular}{|l | l|}
                        \hline
                Motif de réunion & Réunion hebdomadaire \\
                        \hline
                Durée de la réunion & 15min \\
                        \hline
                Lieu & Présentiel \\
                        \hline
                Présent & EM \\
                        & ED \\
                        & TB  \\
                        & BS  \\
                        \hline
                Retard  & / \\
                        \hline
                Absent  & \\
                        \hline
                \end{tabular}


\section{Ordre du jour}
\begin{itemize}
        \item Retour sur la ToDO-List
\end{itemize}

\noindent\rule{\linewidth}{0.5mm} \bigskip
\section{Retour sur la ToDO-List}
\begin{itemize}
        \item Affichage prix : ok
        \item Page shop faite
        \\
        \item Historique achat : ok
        \\
        \item Logo : l'ami d'Ewan ne le fera pas... On va voir avec de la génération IA 
\end{itemize}

\noindent\rule{\linewidth}{0.5mm} \bigskip
\section{ToDO-List}
\subsection*{Frontend}
\begin{itemize}
        \item \textcolor{red}{Gestion des comptes utilisateurs (côté admin)}
        \item \textcolor{red}{Améliorer le header}
        \item \textcolor{blue}{Finir le footer (contact, qu’est ce qu’on dit?)}
        \item \textcolor{blue}{Homepage à mettre en place}
        \item \textcolor{blue}{Page profil}
        \item \textcolor{blue}{Page Admin}
\end{itemize}

\subsection*{Backend}
\begin{itemize}
        \item \textcolor{red}{Gestion des comptes utilisateurs (côté admin)}
        \item \textcolor{darkgreen}{Shop}
        \item \textcolor{darkgreen}{Page profil}
        \item \textcolor{darkgreen}{panier et validation}
        \item \textcolor{blue}{gestion compte : changement mdp, suppression de compte}
\end{itemize}

\subsection*{Autres}
\begin{itemize}
        \item \textcolor{blue}{logo}
        \item \textcolor{blue}{slogan}
        \item \textcolor{blue}{nom}
\end{itemize}

\textcolor{red}{A faire dans la semaine} \\
\textcolor{darkgreen}{Important, mais moins urgent} \\
\textcolor{blue}{Moins urgent}
\\
\bigskip

Petite réunion dûe aux partiels.

\end{document}