\documentclass[10pt,a4paper]{report}
\usepackage{geometry}
\geometry{hmargin=2.5cm,vmargin=1.5cm}

\usepackage{wrapfig}

\usepackage[utf8]{inputenc}
\usepackage{amsmath, amsfonts, amssymb}

\usepackage[dvipsnames]{xcolor}
\colorlet{darkblue}{blue!60!black}

\usepackage{graphicx} %Loading the package
\graphicspath{{data/}} %Setting the graphicspath

\newcommand{\tabitem}{~~\llap{\textbullet}~~}

\usepackage{hyperref}

\usepackage{tcolorbox}
\usepackage{ulem}

\usepackage{titlesec}

\usepackage{hyperref}



\usepackage{subcaption}
\captionsetup{compatibility=false}


\usepackage{array} %%% Permet de faire des tabular mieux
\usepackage{longtable} %%%permet des tableaux sur plusieurs pages


\newcommand{\imp}[0]{$\implies$}

%%%%%%%%%%%%%%%%%%%%%%%%%%%%%%%%%%% mise en page des fonction section, sub, .... %%%%%%%%%%%%%%%%%%%%%%%%%%%%%%%%%%%
\titleformat{\chapter}[display]
{\normalfont\huge\bfseries}{Chapitre \thechapter\ :}{20pt}{\Huge}

\titleformat{\part}[display]
{\Huge\scshape\filright}
{THEME \thepart}
{20pt}
{\thispagestyle{plain}}
\makeatother

\titleformat{\section}
{\normalfont\Large\bfseries}{\thesection}{1em}{}

\titleformat{\subsection}
{\normalfont\large\bfseries}{\thesubsection}{1em}{}

\titleformat{\subsubsection}
{\normalfont\normalsize\bfseries}{\thesubsubsection}{1em}{}

\renewcommand{\thepart}{MO\arabic{chapter}\_\Roman{part}}
\makeatletter

\renewcommand{\thechapter}{MO\arabic{chapter}}
\makeatletter

\renewcommand{\thesection}{\Roman{section}.}

\renewcommand{\thesubsection}{\Roman{section}.\arabic{subsection}.}
\makeatletter

\renewcommand{\thesubsubsection}{MO\arabic{chapter}\_ \Roman{section}.\A{subsection}.\alph{subsubsection}.}
\makeatletter
%%%%%%%%%%%%%%%%%%%%%%%%%%%%%%%%%%%%%%%%%%%%%%%%%%%%%%%%%%%%%%%%%%%%%%%%%%%%%%%%%%%%%%%%%%%%%%%%%%%%%%%%%%%%%%%%%%%%

\newcommand{\linesep}[0]{\begin{center}
                        \rule{0.90\linewidth}{0.15pt}
                        \end{center}}


%\usepackage[titles]{tocloft}
%\renewcommand*\cftchapnumwidth{3em}
%\renewcommand*\cftsecnumwidth{4.3em}
%\renewcommand*\cftsubsecnumwidth{5.3em}


%%%%%%%%%%%%%%%%%%%%%%%%%%%%%%%%%%% Set-up fonction encadrement %%%%%%%%%%%%%%%%%%%%%%%%%%%%%%%%%%%


\newtcolorbox{Encadrer}{colback=white,
colframe=black}

\newtcbox{\encadrer}{colframe=black, colback=white, nobeforeafter,box align=base,size=fbox}

\newtcbox{\encasumup}{colframe=black, colback=white, nobeforeafter,box align=base,size=fbox, left=6pt,right=6pt}

\newtcolorbox{sumup}{colback=white, colframe=black, fonttitle=\bfseries, title={test}}

\usepackage[tikz]{bclogo}
\renewcommand\bcStyleTitre[1]{
\smash{\raisebox{1em}{\colorbox{white}{\color{black}\encasumup{ Résumé }}}}}%

\newtcolorbox{definition}[2][]{%
colback=green!5!white,colframe=green!55!black,fonttitle=\bfseries,
title={Definition} \thetcbcounter: #2,#1}

\newtcolorbox{important}{colback=white,
colframe=red!55!black}


%%%%%%%%%%%%%%%%%%%%%%%%%%%%%%%%%%% Set-up header and foot %%%%%%%%%%%%%%%%%%%%%%%%%%%%%%%%%%%
\usepackage{fancyheadings}
\pagestyle{plain}
%%%%%%%%%%%%%%%%%%%%%%%%%%%%%%%%%%%%%%%%%%%%%%%%%%%%%%%%%%%%%%%%%%%%%%%%%%%%%%%%%%%%%%%%%%%%%%

\title{PPII \\ -- \\ Compte Rendu 2}
\date{12 / 11 / 2023}
\author{SB}

\begin{document}
\maketitle

\tableofcontents \bigskip

\rule{\linewidth}{0.5mm} \bigskip

                \begin{tabular}{|l | l|}
                        \hline
                Motif de reunion & Réunion hebdomadaire \\
                        \hline
                Lieu & Visio (discord) \\
                        \hline
                Présent & EM \\
                        & BS \\
                        & ED \\
                        & TB  \\
                        \hline
                Retard  & / \\
                        \hline
                Absent  & / \\
                        \hline
                \end{tabular}


\section{Ordre du jour}
\begin{itemize}
        \item Frontend
        \item Point sur l'algorithmie
        \item Features page admin / profil
        \item Developpement magasin
\end{itemize}

\rule{\linewidth}{0.5mm} \bigskip
\section{Frontend}
\ \\

Retour sur la page de login et de sign up faites par EM : Tout le monde est en accord avec la "DA", on reste la dessus \\

Une demande de modification : intechanger la section signup et l'image \\

Idée de TB pour inspiration : 
\begin{itemize}
        \item \url{https://www.paris.fr/pages/la-collecte-44}
        \item \url{https://www.grandlyon.com/services/jours-de-collecte-des-dechets }
        \item \url{https://sitetom.syctom-paris.fr/les-dechets/les-categories-de-dechets.html}
\end{itemize}

\uline{Liste pages à faire :} (\sout{deja faite})
\begin{itemize}
        \item Page d'accueil
        \item About us (cf sign up)
        \item \sout{sign up}
        \item \sout{log in}
        \item profil
        \item admin
        \item shop
\end{itemize}

\rule{\linewidth}{0.5mm} \bigskip
\section{Algorithmie}
\ \\

Form : A faire (voir comment gerer les encryptions de passwords) \\

ED : Voir comment recuperer les données d'un form et les mettre dans la BD (avec flask) \\

EM : Debut de l'interface de la BD faite. (stage initial), en cours de réaprentissage de sqlite3 \\

EM, ED : Vori comment faire du multiprocessing en python

Knapsack : fait\ \\

Voyageur de commerce : fait


\rule{\linewidth}{0.5mm} \bigskip
\section{Features}
\subsection{Page profil}
\begin{itemize}
        \item Statistiques (volume ramasser depuis le début, date du derniers ramassge (et historique des ramassages))
        \item statistiques du volume pour chacune des poubelles (carton, recyclage, ordure menagère)
        \item Volume actuelle des poubelles
        \item Suppression du compte, changement mdp, rgpd, request les donnees que possede l'entreprise, ...
        \item Map avec sa / ses poubelles
\end{itemize}

\subsection{Page admin}
\begin{itemize}
        \item Map avec sa poubelle
        \item Statistiques de chacune des poubelles
        \item Date des derniers ramassages
        \item Gestion des profils (afficher les clients, ban, ...)
        \item Afficher la plannification des prochains ramassages (actualisé en "temps réel" par l'algo)
\end{itemize}



\rule{\linewidth}{0.5mm} \bigskip
\section{Developpement magasin}
\begin{itemize}
        \item Mention Geolocalisation 
        \item Une poubelle par type (+ différents volumes, + différentes couleurs) 
        \item Panier (ajout, suppression)
\end{itemize}


\rule{\linewidth}{0.5mm} \bigskip
\section{ToDO-List}
\begin{itemize}
        \item Page sign up : changer image et form de sign up 
        \item créer la DB
        \item TB, BS : apprendre HTML et CSS (flexbox, grid)
        \item ED, EM : voir comment utiliser flask avec les form.
\end{itemize}



\end{document}